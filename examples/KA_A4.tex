\documentclass[chem]{kska}
\title{Klassenarbeit Nr. TEST}

\ihead*{Klasse 1T}
\chead{\thetitle}
\ohead*{01.01.1970}

% TODO
% Varianten (A<->B) testen mit \variant

% OPTIONAL
% Ein Notenschlüssel, leider muss man für jede Note den prozentualen Anteil angeben...
% TODO In eine Funktion packen und mehrere Varianten ablegen.
% TODO Mit Punkten für Kursstufe testen
\DeclareRelGrades{
1       = 0.9506999996 ,
{1,25}  = 0.911533333 ,
{1,5}   = 0.8723666663 ,
{1,75}  =	0.8331999996 ,
2       = 0.794033333 ,
{2,25}	= 0.7548666663 ,
{2,5}   =	0.7156999996 ,
{2,75}	= 0.676533333 ,
3       = 0.6373666663 ,
{3,25}  =	0.5981999996 ,
{3,5}   =	0.559033333 ,
{3,75}  =	0.5198666663 ,
4       = 0.4806999996 ,
{4,25}	= 0.441533333 ,
{4,5}   =	0.4023666663 ,
{4,75}	= 0.3631999996 ,
5       =	0.324033333 ,
{5,25}  = 0.2848666663 ,
{5,5}   =	0.2456999996 ,
{5,75}  =	0.206533333 ,
6   =	0.1673666663
}


\begin{document}
  % Die Linie für den Namen
  \nameline
  
  \begin{framed}\noindent Lies dir die Aufgaben \emph{genau} durch bevor du mit der Beantwortung
beginnst! Stelle deine Antworten sauber und übersichtlich dar.\\
Strukturformeln müssen nur angegeben werden, wenn dies in der
Aufgabestellung ausdrücklich verlangt wird.
\end{framed}


  \begin{question}[subtitle=Wiederholungsaufgabe]{2}
    Berechne.
    \begin{tasks}(3)
      \task $1 + 1$
      \task $1 \cdot 1$
      \task $1 : 1$
    \end{tasks}
  \end{question}
  \begin{solution}
    \begin{tasks}(3)
      \task $1 + 1 = 2$
      \task $1 \cdot 1 = 1$
      \task $1 : 1 = 1$
    \end{tasks}
  \end{solution}
  
  
  \begin{question}{4}
    \begin{tasks}(1)
      \task  Eine Teilaufgabe mit einem Punkt. (\points{1})
      \task  Eine Teilaufgabe mit drei Punkten. (\points{3})
    \end{tasks}
  \end{question}
  \begin{solution}
      \begin{tasks}(1)
      \task  Eine Teilaufgabe mit einem Punkt. (\points{1})\\
        \itshape{Diese Aufgabe hat eine kurze Lösung.}
      \task  Eine Teilaufgabe mit drei Punkten. (\points{3})\\
        \itshape{Bei drei Punkten muss man schon etwas mehr schreiben. Aber nur etwas.}
    \end{tasks}
  \end{solution}
  
  
  \begin{question}[skip-below=3cm]{2}
  Eine Aufgabe mit \SI{3}{\centi\meter} Platz darunter, um die Antwort(en) eintragen zu können.
  \end{question}
  \begin{solution}
    \itshape{Schreib doch was du willst.}
  \end{solution}
  
  \begin{question}
    Vervollständige folgende Reaktionsgleichung:
    \begin{reaction*}
      \qquad CH4 + \qquad O2 -> \qquad CO2 + \qquad H2O
    \end{reaction*}
  \end{question}
  \begin{solution}
    Vervollständige folgende Reaktionsgleichung:
    \begin{reaction*}
      1 CH4 + 2 O2 -> 1 CO2 + 2 H2O
    \end{reaction*}
  \end{solution}

  % Vertikalen Raum auffüllen, so dass die Box für die Note unten steht
  \vfill
  
  % Selbsteinschätzung Unterrichtsnote
  \selfeval
  % Box für Note(n)
  \markingbox
  
  \newpage
  
  \title{Lösungen}
  \maketitle
  % Gibt die oben definierten Lösungen aus
  \printsolutions
  
  \vfill
  
  % OPTIONAL
  \section*{Notenschlüssel}
  \footnotesize\setlength\tabcolsep{2pt}
  \begin{tabularx}{\linewidth}{|r|*{21}{Z|}}
  \toprule
  Punkte
  & $\grade*{1}$ & $\grade*{1,25}$ & $\grade*{1,5}$ & $\grade*{1,75}$
  & $\grade*{2}$ & $\grade*{2,25}$ & $\grade*{2,5}$ & $\grade*{2,75}$
  & $\grade*{3}$ & $\grade*{3,25}$ & $\grade*{3,5}$ & $\grade*{3,75}$
  & $\grade*{4}$ & $\grade*{4,25}$ & $\grade*{4,5}$ & $\grade*{4,75}$
  & $\grade*{5}$ & $\grade*{5,25}$ & $\grade*{5,5}$ & $\grade*{5,75}$
  & $\grade*{6}$ \\
  \midrule
  Note
  & 1 & 1-- & 1--2 & 2+ 
  & 2 & 2-- & 2--3 & 3+
  & 3 & 3-- & 3--4 & 4+
  & 4 & 4-- & 4--5 & 5+
  & 5 & 5-- & 5--6 & 6+
  & 6 \\
  \bottomrule
  \end{tabularx}
  
  
\end{document}
